\documentclass[default,pdf,colorBG,slideColor]{prosper}
\hypersetup{pdfpagemode=FullScreen}

\title{Anonymous Communication}
\subtitle{An Introduction}
\author{James Lee}
\email{jlee23@umbc.edu}
\institution{University of Maryland, Baltimore County}

\begin{document}
\maketitle

\begin{slide}{Outline}
\begin{itemize}
\item Introduction
\item Relays
\item Mixes
\item Questions
\item References
\end{itemize}
\end{slide}

\begin{slide}{Why Is Anonymity Important?}
\begin{itemize}
\item A person wants to protect their privacy.
\item Fear of retaliation, discrediting, unpopular sentiment, etc.
\item Examples:
\begin{itemize}
	\item Crime tip lines
	\item Political discussion, voting
	\item P2P file sharing
\end{itemize}
\item Users are identified by unique addresses (IP, MAC, etc.)
\end{itemize}
\end{slide}

\begin{slide}{What Is the Goal?}
\begin{itemize}
\item Make it appear to an outside observer that no communication happened at all.
\item However, on the Internet there is spyware, untrusted routers, packet sniffers, trojans, wire tappers.
\item Strive for sender anonymity, receiver anonymity, sender-receiver unlinkablity.
\end{itemize}

\begin{description}
\item{Anonymity} is the state of being not identifiable within a set of subjects.
\item{Unlinkability} means an attacker would not be able to relate two or more subjects by observing the system.
\end{description}
\end{slide}

\begin{slide}{Trusted and Semi-trusted Relays}
Relays rely on one central trusted node to provide security.
\begin{itemize}
\item The Anon.penet.fi relay
\item Anonymizer and SafeWeb
\item Type I remailers
\item Crowds
\item Nym servers
\end{itemize}
\end{slide}

\begin{slide}{The Anon.penet.fi Relay}
\begin{itemize}
\item Pseudoanonymous email service started in 1993.
\item Kept table mapping pseudonyms to real email addresses.
\item Email to a pseudonym would be forwarded to the real address.
\item Email from a pseudonym would be stripped of all identifying information and relayed.
\item Service shut down in 1996 after being forced to reveal the identity of a user in a copyright trial.
\end{itemize}
\end{slide}

\begin{slide}{Anonymizer and SafeWeb}
\begin{itemize}
\item Anonymizer and SafeWeb are both commercial services which offered web proxies.
\item Filters out or wraps active content like JavaScript or Java which could be used to identify users.
\item Like anon.penet.fi, anonymity depends on the integrity of the company providing the service.
\item Less vulnerable to legal attacks since logs don't have to be kept.
\end{itemize}
\end{slide}

\begin{slide}{Type I Remailers}
\begin{itemize}
\item Relay email after stripping all identifying information.
\item Many implementations allowed remailers to be chained together.
\item Reply information encrypted within the message itself.
\item Unlike anon.penet.fi, no database of pseudonyms is kept, but each relay node still has to be trusted.
\end{itemize}
\end{slide}

\begin{slide}{Crowds}
\begin{itemize}
\item User downloads a list of participants from a central server.
\item User then relays a web request to a randomly selected node in the crowd.
\item Any received request is either relayed to another random node or sent to the final recipient randomly.
\item Cannot tell if a request was initiated by the previous node or if it was just passing it on.
\item Dishonest nodes can collude to discover identity of users.
\end{itemize}
\end{slide}

\begin{slide}{Nym Servers}
\begin{itemize}
\item Like remailers, but also give users pseudonyms.
\item Routing information encoded within message.
\item Nym servers don't have to be trusted since they can only determine the location of other nym servers.
\end{itemize}
\end{slide}

\begin{slide}{Mix Systems}
These systems use techniques from remailers plus cryptography to ensure anonymity.
\begin{itemize}
\item Chaum's original mix
\item ISDN mixes, real time mixes, and web mixes
\item Onion Routing
\item Tor
\end{itemize}
\end{slide}

\begin{slide}{Chaum's Original Mix}
\begin{itemize}
\item A ``mix'' node hides the correspondence between its input messages and output messages using cryptography.
\item Messages to be anonymized are encrypted and relayed through a mix which has a well-known public key.
\item The mix decrypts the message, strips out identifying information, adds random bits (junk) to the end passes it on to the recipient.
\item Mixes also mix together many messages, sending messages out in a different order than they were received.
\end{itemize}
\end{slide}

\begin{slide}{ISDN Mixes}
\begin{itemize}
\item Designed for constant, high traffic such as streaming data or voice.
\item Uses a cascade of mixes with a persistent route set up before sending data.
\item Does not mix message order, uses stream cipher instead of block cipher.
\item Based on ISDN infrastructure; not well suited for TCP/IP.
\end{itemize}
\end{slide}

\begin{slide}{Onion Routing}
\begin{itemize}
\item Like ISDN mixes in that a route is established before sending messages.
\item First message is encrypted in layers that can only be decrypted by a chain of onion routers.
\item First message contains keys for the routers for encryption of subsequent messages.
\item Vulnerable to ``timing attacks.''  Adversary can watch the patterns of traffic moving between routers to identify users.
\end{itemize}
\end{slide}

\begin{slide}{Tor}
\begin{itemize}
\item An implementation of onion routing for arbitrary TCP streams.
\item Any application can route through the Tor network without modification using its SOCKS proxy service.
\item Also provides mechanisms for ``hidden servers.'' 
\end{itemize}
\end{slide}

\begin{slide}{Questions}

\end{slide}

\begin{slide}{References}
\begin{itemize}
\item George Danezis and Claudia Diaz. A Survey of Anonymous Communication Channels. Submitted to the \emph{Journal of Privacy Technology}, 40 pages, 2006.

\item Andy Jones. Anonymous Communication on the Internet. http://www10.cs.rose-hulman.edu/Papers/Jones.pdf. Retrieved Feb 17, 2008.
\end{itemize}
\end{slide}
\end{document}
