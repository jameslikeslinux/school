\documentclass{beamer}

\usepackage{beamerthemesplit}

\title{TorConnect}
\subtitle{An Anonymous P2P Service}
\author{James Lee}
\institute{University of Maryland, Baltimore County}
\date{May 21, 2008}

\begin{document}

\begin{frame}
\titlepage
\end{frame}

\section{Introduction}

\subsection{Motivation}
\begin{frame}{Motivation}
\begin{center}
\includegraphics[width=.9\textwidth]{vfv-crowd.jpg}
\end{center}

\emph{I want to be able to participate in a community anonymously.}

\begin{itemize}
\pause
\item Why would anyone want to remain anonymous?
% Have illegal material.
% Have embarrassing material.
% Fear retribution for sensitive, offensive message.
% Voice opinion.
% Government censorship.

\pause
\item Why participate in a community?
% Meet others with a shared interest.
% Develop friendships. 
% Example: UMBC Hub.

\pause
\item How can we build a new community?
% Must have ability to discover new users.
% Users need reason to join.
% Could be filesharing and chatting.
% UMBC hub not as popular without filesharing.
% Must also be easy to use.

\pause
\item Why use Tor?
% Persistent routes->reduce latency->increasing throughput.
% Largest anonymity network.
% Academically scrutinized.
% Easy application interface with SOCKS and control protocol.
\end{itemize}
\end{frame}

\subsection{Previous Work}
\begin{frame}{Previous Work}
Most work on Tor attempts to make it easier for a single user to access the network.

\begin{columns}
\column{.3\textwidth}
\begin{itemize}
% Examples:
\item<2-> Torbutton, Vidalia
% Torbutton for Firefox.
% Vidalia for generic communication.
% Easy to install; almost no configuration.
% Method for actual communication must be provided some other way.

\item<3-> TorChat
% Clients act like hidden services->16 char pseudonym.
% Chat by connecting to other user by pseudonym.
% Decentralized->not suitable for community. (must already know addr)
% Setup is easy on Windows.
% Manual setup on Linux.

\pause
\item<4-> Direct Connect
% Mention problems.
% Esp. with UMBC hub.


%
% Segue
%
\end{itemize}

\column{.7\textwidth}
\includegraphics<2>[width=\textwidth]{torbutton.png}
\includegraphics<3>[width=\textwidth]{torchat.png}
\includegraphics<4>[width=\textwidth]{dc.png}
\end{columns}
\end{frame}

\section{Design}
\subsection{Handshake}

\begin{frame}{Source Address Discovery}
\begin{itemize}
\item Current P2P systems are designed around IP addressing.
% Example: BitTorrent

\pause
\item Servers can reliably get a client's address from TCP.

\pause
\item This doesn't work on Tor.  How can a Tor service get clients' addresses (psedonyms)?

\pause
\item Just have the client tell the server who they are.

\pause
\item Handshake prevents clients from lying.
\end{itemize}
\end{frame}

\begin{frame}{The Handshaking Process}
\begin{center}
\includegraphics[width=.9\textwidth]{handshake.pdf}
\end{center}
\end{frame}

\subsection{Protocol}
\begin{frame}{The Thing About the Protocol}
After a handshake, you can treat the connection like any other TCP stream.  So you adapt just about any TCP protocol.
% Proposal called for identifying tokens in every message instead of handshakes.
% After handshaking, any protocol can be used, just replace IP addresses with Tor pseudonyms where applicable.
% Made protocol design a boring exercise.
% Did my best to make it simple to parse and implement and prove my concept in a short period of time.
% Instead focused on robust, generic Tor control, connection libraries which can be used by more mature protocols.
\end{frame}

\subsection{Application}
\begin{frame}{Application Components}
\begin{center}
\includegraphics[width=.9\textwidth]{architecture.pdf}
\end{center}
\end{frame}

\appendix
\section{References}
~\nocite{tor-design}
\bibliographystyle{plainurl}
\bibliography{final-presentation}

\end{document}
